\documentclass[margin,line]{res}

\usepackage{url}
\usepackage{comment}

%Curriculum Vitae. Please submit your curriculum vitae by uploading a plain text or PDF file that should include:
% 	
%
%    * publications in computer science.
%    * professional experience (including fellowships, teaching, teaching assistantships, etc.); include position, approximate dates, duties and name of supervisor.
%    * honors awarded for scholarship (honorary society memberships, prizes, scholarships, etc.)
%    * other experience or courses you have taken (in computer science, engineering,mathematics, etc.) that you think particularly qualify you for graduate work in computer science.

\oddsidemargin -.5in
\evensidemargin -.5in
\textwidth=6.0in
\itemsep=0in
\parsep=0in

\newenvironment{list1}{
  \begin{list}{\ding{113}}{%
      \setlength{\itemsep}{0in}
      \setlength{\parsep}{0in} \setlength{\parskip}{0in}
      \setlength{\topsep}{0in} \setlength{\partopsep}{0in} 
      \setlength{\leftmargin}{0.17in}}}{\end{list}}
\newenvironment{list2}{
  \begin{list}{$\bullet$}{%
      \setlength{\itemsep}{0in}
      \setlength{\parsep}{0in} \setlength{\parskip}{0in}
      \setlength{\topsep}{0in} \setlength{\partopsep}{0in} 
      \setlength{\leftmargin}{0.2in}}}{\end{list}}


\usepackage{marvosym} % For cool symbols.
\usepackage{hyperref}

%\begin{document}


\begin{document}

\name{Alan Ritter \vspace*{.1in}}
\address{
\begin{tabular}{l}
Postdoctoral Fellow \\
Machine Learning Department \\
Carnegie Mellon University \\
\end{tabular}
}

\begin{resume}

\section{\sc Contact Information}
\vspace{.05in}
\begin{tabular}{lp{4in}}
{\it E-mail:} & rittera@cs.cmu.edu \\       
{\it Office:} & 8004 GHC \\
{\it Phone:}  &(360) 820-1090 \\       
{\it WWW:} & http://aritter.github.io/ \\     
\end{tabular}

\section{\sc Research Interests}
%Much of humanity's collective knowledge is encoded in ever expanding volumes of text. Broadly I am interested in getting computers to understand natural language at scale, and new applications this will enable. 
%I enjoy working with large quantities of text which is not restricted to a narrow domain, such as that found on the {\bf Web} or in {\bf Social Media}.

Natural Language Processing in Informal Text, 
Latent Variable Models of Lexical Semantics, Machine Reading, Text Mining,
Intelligent User Interfaces. 


\section{\sc Education}
{\bf University of Washington}, Seattle, WA USA\\
\vspace*{-.1in}
\begin{list1}
\item[] Ph.D. in Computer Science \& Engineering
\begin{list2}
\vspace*{.05in}
\item Thesis Topic: Information Extraction in Social Media
\item Fall 2007 - Summer 2013
\end{list2}
\end{list1}
{\bf Western Washington University}, Bellingham, WA USA\\
\vspace*{-.1in}
\begin{list1}
\item[] B.S./M.S., Computer Science,  December 2006
  \begin{list2}
    \vspace*{.05in}
    \item Overall GPA: 4.00
  \end{list2}
\end{list1}

%\section{\sc Professional Memberships}

%International Society for Computers and their Applications
%\vspace*{-2.5mm}

%Association for Computing Machinery
%\vspace*{-2.5mm}

\section{\sc Awards}

Facebook Fellowship Finalist (2012)
\vspace*{-2.5mm}

EMNLP Best Reviewer Awards (2011,2012)
\vspace*{-2.5mm}

Best Student Paper Award for \emph{Learning to Generalize for Complex Selection Tasks} at IUI (2009)
\vspace*{-2.5mm}

NDSEG Fellowship (2008)
\vspace*{-2.5mm}

WWU Outstanding Graduating Senior in Computer Science (2006)

%Graduated Magna Cum Laude
%\vspace*{-2.5mm}

%Departmental Tuition Waiver (2003-2004)
%\vspace*{-2.5mm}

%ASEE/ONR NREIP Summer Research Internship (2006)
%\vspace*{-2.5mm}

%Google \emph{Summer of Code} Grant (2005)
%\vspace*{-2.5mm}

%Golden Key International Honor Society
%\vspace*{-2.5mm}

%Phi Kappa Phi Honor Society

\begin{comment}
\section{\sc Research Projects}

{\bf Information Extraction in Social Media}
Off-the shelf tools such as Part of Speech Taggers and Named Entity Recognizers perform poorly when applied to Social Media text due to its noisy and unique style. To address this I have been working towards building a set of Twitter-specific text processing tools [EMNLP 2011a].  Users of Social Media sites frequently discuss events which will occur in the future. By annotating Named Entities and resolving temporal expressions (for example ``next Friday''), we are able to automatically extract a calendar of popular events occurring in the near future from Twitter [KDD 2012].

{\bf Conversational Modeling in Social Media}
I have worked on unsupervised modeling of dialogue acts in Twitter [NAACL 2010]. By remaining agnostic about the set of classes, we are able to learn a model which provides insight into the nature of communication in a new medium.  I have investigated the feasibility of automatically replying to status messages by adapting techniques from Statistical Machine Translation [EMNLP 2011b], using millions of naturally occurring Twitter conversations as parallel text. Although there are many differences between conversation and translation, with a few conversation-specific adaptations we are able to build response models which often generate appropriate replies to Twitter status posts. This work has several possible applications, including conversationally aware predictive text entry.

{\bf Latent Variable Models of Lexical Semantics}
I have applied a variant of Latent Dirichlet Allocation to automatically infer the argument types or Selectional Preferences of textual relations [ACL 2010].  Our model of selectional preferences is useful in filtering improper applications of inference rules in context, showing a substantial improvement over a state-of-the-art rule-filtering system which makes use of a predefined set of classes.  In addition, I have applied latent variable models to automatically induce an appropriate set of categories for events mentioned on Twitter [KDD 2012].  By leveraging large quantities of unlabeled data we are able to outperform a supervised baseline at the task of categorizing events extracted from Twitter.  I have also investigated an approach to Distant Supervision using Topic Models.  As a distant source of supervision we make use of lexical entries from Freebase, a large, open-domain database, to generate constraints in the model. This approach leverages the ambiguous supervision provided by Freebase in a principled way, significantly outperforming Co-Training on a weakly supervised named entity classification task [EMNLP 2011a].
\end{comment}

\section{\sc Professional Experience}

{\bf The Ohio State University},
Columbus OH USA

\vspace{-.3cm}
{\em Assistant Professor - Computer Science and Engineering} \hfill {\bf August 2014 - ...}\\

{\bf Carnegie Mellon University},
Pittsburgh PA USA

\vspace{-.3cm}
{\em Postdoctoral Fellow - Machine Learning Department} \hfill {\bf August 2013 - August 2014}\\
Supervisor: Tom Mitchell

{\bf Microsoft Research},
Redmond WA USA

\vspace{-.3cm}
{\em Consultant - Natural Language Processing Group} \hfill {\bf July 2013 - August 2013}\\

{\bf Microsoft Research},
Redmond WA USA

\vspace{-.3cm}
{\em Intern - Natural Language Processing Group} \hfill {\bf June 2009 - September 2009}\\
%Investigated conversations on Twitter.
Crawled Twitter to gather a corpus of conversations.
Designed a generative model of dialogue, and conducted experiments to demonstrate
the feasibility of unsupervised modeling of dialogue acts in new media.  
Investigated data-driven response generation in Social media.
This work resulted  in publication at [NAACL 2010] and [EMNLP 2011a].\\
Supervisors: Colin Cherry, Bill Dolan

{\bf Microsoft Research},
Redmond WA USA

\vspace{-.3cm}
{\em Intern - Knowledge Tools Group (Machine Learning Department)} \hfill {\bf June 2008 - August 2008}\\
Investigated new methods for Smart Selection in the file browser domain.  
Proposed the idea of {\em Learning to Generalize} in Interactive Machine
Learning.  Implemented a prototype {\em Smart File Browser}, and conducted User Studies.
This work resulted in an award paper at [IUI 2009].\\
Supervisor: Sumit Basu

%\newpage

{\bf University of Washington},
Seattle WA USA

\vspace{-.3cm}
{\em Software Engineer} \hfill {\bf April 2007 - September 2007}\\
Implemented updates and maintained {\sc TextRunner}, an Open IE based search
engine.  Developed a web-scale hypernym extractor.\\
Supervisors: Stephen Soderland, Oren Etzioni

{\bf Microsoft}, 
Redmond WA USA

\vspace{-.3cm}
{\em Software Design Engineer} \hfill {\bf January 2007 - March 2007}\\
Windows Media DRM group.

{\bf Naval Research Laboratory, Marine Meteorology Division},
Monterey, CA USA

\vspace{-.3cm}
{\em Intern - Data Assimilation Group} \hfill {\bf June 2006 - August 2006}\\
Used TAU (Tuning and Analysis Utilities) to instrument, profile, analyze,
and improve scalability of research and operational NRL Data Assimilation 
Fortran/MPI code.  Taught a class on automatic instrumentation of parallel 
programs for NRL scientists.  Presented a Seminar on scalability enhancements
to NAVDAS.\\
Supervisor: William F. Campbell

{\bf The NetBSD Project}

\vspace{-.3cm}
{\em Google ``Summer of Code'' Grant} \hfill {\bf June 2005 - August 2005}\\
Ported FreeBSD's NDIS Compatibility Layer to NetBSD. \\
Mentor: Philip A. Nelson

\section{\sc Teaching Experience}

{\bf University of Washington}, Seattle WA USA\\
{\em Teaching Assistant - Machine Learning} \hfill {\bf Jan 2009 - March 2009, Jan 2011 - March 2011} \\
\url{http://www.cs.washington.edu/education/courses/cse446/} \\
Helped to design and grade homework assignments, answered student questions one-on-one, prepared and presented an hour long
lecture on Neural Networks. \\
Supervisor: Oren Etzioni

\begin{comment}
\section{\sc Mentoring}

{\em Sam Clark} \hfill {\bf March 2010 - June 2011} \\
Supervised Sam's work annotating a corpus of Tweets with Part of Speech (POS) and Chunk tags, and building a POS tagger and Chunker
for Tweets.  Sam is now working at a Seattle startup, {\em Decide.com}
\end{comment}

\section{\sc Patents}
US Patent Application US20110010669 \\
``Items Selection Via Automatic Generalization''

\section{\sc Invited Talks}
``Extracting Knowledge from Informal Text'' \\
University of Pittsburgh, January 2014 \\
Machine Learning Lunch, Carnegie Mellon University, October 2013 \\
IEEE Intelligence and Security Informatics, June 2013 \\
Toyota Technological Institute at Chicago, April 2013 \\
Microsoft Research, April 2013 \\
University of Texas at Dallas, March 2013 \\
The Ohio State University, March 2013 \\
Machine Learning and Friends Lunch, University of Massachusetts Amherst, March 2013 \\
Human Language Technology Center of Excellence, March 2013 \\

``Modeling Conversations in Social Media'' \\
Microsoft Research, September 2011

%\newpage
``Extracting a Calendar from Twitter'' \\
Pacific Northwest National Laboratory, January 2013 \\
Proteus Group, New York University, September 2012 \\
Wavii, April 2012 \\
Facebook, March 2012 \\
Google Research, September 2011 \\
Yahoo! Labs, September 2011 \\
Twitter, May 2011 \\
Information Sciences Institute, University of Southern California, February 2011 \\
%UW CSE Industrial Affiliates Meeting, October 2011

%``What Can Machines Learn by Reading Tweets?'' \\
%ICT workshop on Reasoning with Text - \url{http://projects.ict.usc.edu/rwt2011/} \\
%February 2011


%``Learning to Generalize for Complex Selection Tasks'' \\
%UW CSE Industrial Affiliates Meeting \\
%October 2009
%
%``Modeling Conversations on Twitter''\\
%UW/Microsoft Quarterly Symposium on Computational Linguistics \\
%October 2009
%
``Finding Contradictions in Text Using Functions'' \\
Boeing Research \\
February 2008

\section{\sc Press}
``Forget today's news: Zapaday will break future news''\\
\url{http://thenextweb.com/media/2012/05/01/forget-todays-news-zapaday-will-break-future-news/}
The Next Web \\
May 2012

``'Hello, computer!' UW prof and students search outside the box''\\
\url{http://www.geekwire.com/2011/computer-uw-prof-students-search-box/} \\
GeekWire \\
November 2011

\section{\sc Tutorials}
``Latent Variable Models of Lexical Semantics'' \\
Yahoo! Machine Learning Seminar (@ UW) \\
January 2012


\begin{comment}
``Profiling NAVDAS'' \\
Naval Research Laboratory Monterey, CA \\
August 2006

``Profiling Parallel Fortran/MPI programs using TAU'' (Training Class for NRL Scientists) \\
Naval Research Laboratory Monterey, CA \\
August 2006
\end{comment}

\section{\sc Service}
Area Chair \\
\emph{ACL 2014 (NLP for the web and social media), NAACL 2015 (Information Extraction and Question Answering)}

Co-Organizer \\
\emph{Sentiment Analysis in Twitter Shared Task (SemEval 2013,2014,2015)}

Session Chair \\
\emph{EMNLP 2012 (Social Media)}, \emph{NAACL 2013 (Social Media)}, \emph{ACL 2014 (Social Media)}

Journal Reviewing \\
\emph{ACM Computing Surveys (2014), TACL (2013,2014), Computational Linguistics (2013), JAIR (2012), TOIS (2012), TKDE (2014), JMLR MLOSS (2013), Journal of Natural Language Engineering (2014)}

Program Committee \\
\emph{EMNLP (2010,2011,2012,2013,2014), ACL (2012,2013), NAACL-HLT (2012,2013), CoNLL (2013,2014), AAAI (2012), EACL (2012,2014), Coling (2014), ICWSM (2012), LREC (2012), IJCAI (2011), *SEM (2013), WWW (2014,2015), CIKM (2014)}

Workshop Reviewing \\
\emph{``@NLP can u tag \#user\_generated\_content ?!'' (LREC 2012), Workshop on Language in Social Media (2011, 2012, 2013, 2014), Workshop on Analyzing Microtext (2011,2013), Workshop on Relational Models of Semantics (2011), Workshop on Information Extraction and Knowledge Acquisition (2011), ACL Student Research Workshop (2012), AKBC-WEKEX (2012,2013), CIKM workshop on Politics, Elections and Data (2012,2013), TextGraphs 2013, Workshop on Latent Attribute Prediction in Social Media (ACL 2014)}

%Supervision Committee \\
%\emph{Concept Extraction Challenge at the Workshop on Making Sense in Microposts (WWW 2013)}

\begin{comment}
External Reviewer \\
\emph{IUI (2013), ACL (2011), AAAI (2010), UIST (2009)}
\end{comment}

\begin{comment}
Reviewer \emph{JAIR  (2012)}
\vspace*{-2.5mm}

Reviewer \emph{ACM Transactions on Information Systems (2012)}
\vspace*{-2.5mm}

Program Committee \emph{EMNLP-CoNLL 2012}
\vspace*{-2.5mm}

Program Committee \emph{AAAI Spring Symposium on Analyzing Microtext 2013}
\vspace*{-2.5mm}

Program Committee \emph{ACL 2012}
\vspace*{-2.5mm}

Program Committee \emph{NAACL-HLT 2012}
\vspace*{-2.5mm}

Program Committee \emph{AAAI 2012}
\vspace*{-2.5mm}

Program Committee \emph{EACL 2012}
\vspace*{-2.5mm}

Program Committee \emph{ICWSM 2012}
\vspace*{-2.5mm}

Scientific Committee \emph{LREC 2012}
\vspace*{-2.5mm}

Program Committee \emph{Workshop on Language in Social Media 2012}
\vspace*{-2.5mm}

Program Committee \emph{EMNLP 2011}
\vspace*{-2.5mm}

Reviewer \emph{ACL 2011}
\vspace*{-2.5mm}

Program Committee \emph{IJCAI 2011}
\vspace*{-2.5mm}

Program Committee \emph{Workshop on Language in Social Media 2011}
\vspace*{-2.5mm}

Program Committee \emph{Workshop on Analyzing Microtext 2011}
\vspace*{-2.5mm}

Program Committee \emph{Workshop on Relational Models of Semantics 2011}
\vspace*{-2.5mm}

Program Committee \emph{Workshop on Information Extraction and Knowledge Acquisition 2011}
\vspace*{-2.5mm}

Program Committee \emph{EMNLP 2010}
\vspace*{-2.5mm}

Reviewer \emph{AAAI 2010}
\vspace*{-2.5mm}

Reviewer \emph{UIST 2009}
\vspace*{-2.5mm}

Prospective Graduate Student Committee, \emph{University of Washington} 2010
\end{comment}

%\newpage
\section{\sc Publications}


Alan Ritter, Luke Zettlemoyer, Mausam and Oren Etzioni \\
``Modeling Missing Data in Distant Supervision for Information Extraction''\\
Proceedings of TACL 2013

Alan Ritter, Mausam, Oren Etzioni, Sam Clark \\
``Open Domain Event Extraction from Twitter''\\
Proceedings of KDD 2012

Wei Xu, Alan Ritter, Bill Dolan, Ralph Grishman, Colin Cherry \\
``Paraphrasing for Style''\\
Proceedings of Coling 2012

Alan Ritter, Sam Clark, Mausam, Oren Etzioni \\
``Named Entity Recognition in Tweets: An Experimental Study''\\
Proceedings of EMNLP 2011

Alan Ritter, Colin Cherry, Bill Dolan \\
``Data-Driven Response Generation in Social Media'' \\
Proceedings of EMNLP 2011

Alan Ritter, Mausam and Oren Etzioni \\
``A Latent Dirichlet Allocation Method for Selectional Preferences'' \\
Proceedings of ACL 2010

Alan Ritter, Colin Cherry and Bill Dolan \\
``Unsupervised Modeling of Twitter Conversations'' \\
Proceedings of HLT-NAACL 2010

Alan Ritter and Sumit Basu \\
``Learning to Generalize for Complex Selection Tasks'' \\
{\bf Best Student Paper Award} \\
Proceedings of IUI 2009

Alan Ritter, Stephen Soderland and Oren Etzioni \\
``What is this Anyway?  Automatic Hypernym Discovery'' \\
Proceedings of the AAAI 2009 Spring Symposium on Learning by Reading and Learning to Read

Alan Ritter, Doug Downey, Stephen Soderland, and Oren Etzioni \\
``It's a Contradiction -- No, It's Not: A Case Study using Functional Relations'' \\
Proceedings of EMNLP 2008

Alan Ritter, James W. Hearne, Philip A. Nelson \\
``Distributional Word Clustering in Parallel'' \\
ISCA PDCS 2006

J. Michael Meehan, Alan L. Ritter \\
``Machine Learning Approach to Tuning Distributed Operating System Load Balancing Algorithms'' \\
ISCA PDCS 2006

Alan Ritter \\
``NDIS Network Driver'' \\
Dr. Dobb's Journal \\
January 2006

\begin{comment}
\section{\sc Skills} 
\begin{list2}
\item Languages:  C, C++, C\#, Java, Perl, Python, bash, Lisp, R, x86 assembly, Fortran, etc\ldots
\item APIs: MPI, POSIX, Win32, .NET, Lucene, etc\ldots
\end{list2}
\end{comment}

\newpage

\begin{comment}
\section{\sc References}
\begin{tabular}{lr}
% Referee 1
\begin{minipage}[t]{2.5in}
Sumit Basu\\
Researcher\\
Machine Learning Department\\
Microsoft Research\\
\Telefon\ 425 706 7971\\
\Letter\ \href{mailto:sumitb@microsoft.com}{sumitb\textrm{@}microsoft.com}
\end{minipage}
&
% Referee 2
\begin{minipage}[t]{2.5in}
Bill Dolan\\
Research Manager\\
Natural Language Processing Group\\
Microsoft Research\\
\Telefon\ 425 706 3709\\
\Letter\ \href{mailto:billdol@microsoft.com}{billdol\textrm{@}microsoft.com}
\end{minipage}

\\
\\ % Additional newline for spacing.
% Referee 3
\begin{minipage}[t]{2.5in}
Oren Etzioni\\
WRF Entrepreneurship Professor\\
Computer Science and Engineering\\
University of Washington\\
\Telefon\ 206 685 3035\\
\Letter\ \href{mailto:etzioni@cs.washington.edu}{etzioni\textrm{@}cs.washington.edu }
\end{minipage}
&
\begin{minipage}[t]{2.5in}
Mausam\\
Research Assistant Professor\\
Computer Science and Engineering\\
Univeristy of Washington\\
\Telefon\ 206 685 1964\\
\Letter\ \href{mailto:mausam@cs.washington.edu}{mausam\textrm{@}cs.washington.edu}
\end{minipage}
\\
\\
\begin{minipage}[t]{2.5in}
Luke Zettlemoyer\\
Assistant Professor\\
Computer Science and Engineering\\
University of Washington\\
\Telefon\ 206 685 1227\\
\Letter\ \href{mailto:lsz@cs.washington.edu}{lsz\textrm{@}cs.washington.edu}
\end{minipage}

\end{tabular}
\end{comment}

\end{resume}

\end{document}
